\documentclass[12pt]{article}

\usepackage[english]{babel}
\usepackage{blindtext}
\usepackage{graphicx}
\usepackage{amssymb}
\usepackage{amsthm}
\usepackage{amsmath}
\usepackage{bbold}
\usepackage{bm}
\usepackage{physics}
\usepackage[a4paper, width = 185mm, top = 15mm, bottom = 15mm]{geometry}
\usepackage{calligra}
\usepackage{float}

\DeclareMathAlphabet{\mathcalligra}{T1}{calliga}{m}{n}
\DeclareFontShape{T1}{calligra}{m}{n}{<->s*[2.2]callig15}{}

\newcommand{\scripty}[1]{\ensuremath{\mathcalligra{#1}}}
\renewcommand* \d{\mathop{}\!\mathrm{d}}



\begin{document}
\noindent
$$W = (I-VP)^{-1}V$$
with
$$V = \begin{pmatrix}
\tilde{V} & V\\
V & \tilde{V}
\end{pmatrix}\text{ and } P = \begin{pmatrix}
P^\uparrow & 0\\
0 & P^\downarrow
\end{pmatrix}$$
So we have:
$$I-VP=\begin{pmatrix}
I - \tilde{V}P^\uparrow & -VP^\downarrow\\
-VP^\uparrow & I-\tilde{V}P^\downarrow
\end{pmatrix}$$
The inverse is then given by:
$$(I-VP)^{-1}=\begin{pmatrix}
(I-\tilde{V}P^\uparrow)^{-1} + (I-\tilde{V}P^\uparrow)^{-1}VP^\downarrow SVP^\uparrow(I-\tilde{V}P^\uparrow)^{-1} & (I-\tilde{V}P^\uparrow)^{-1}VP^\downarrow S\\
SVP^\uparrow(I-\tilde{V}P^\uparrow)^{-1} & S
\end{pmatrix}$$
With $S=(I-\tilde{V}P^\downarrow-VP^\uparrow (I-\tilde{V}P^\uparrow)^{-1}VP^\downarrow)^{-1}$
So we have:
\begin{align*}
W^\uparrow &= ((I-\tilde{V}P^\uparrow)^{-1} + (I-\tilde{V}P^\uparrow)^{-1}VP^\downarrow SVP^\uparrow(I-\tilde{V}P^\uparrow)^{-1} )\tilde{V} + (I-\tilde{V}P^\uparrow)^{-1}VP^\downarrow S V\\
W^\downarrow &= SVP^\uparrow(I-\tilde{V}P^\uparrow)^{-1}V + S\tilde{V}
\end{align*}


\end{document}